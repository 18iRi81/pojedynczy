\section{Preambuła}

Typowa preambuła wygląda następująco:
\begin{verbatim}
\documentclass[wmii,inf,mgr]{uwmthesis}

\date{2015}
\title{Wykorzystanie Klasy Dokumentu 
  do Pisania Prac Dyplomowych}
\author{Aleksander Denisiuk}
\etitle{Using of the Document Class for Theses}
\wykonanaw{katedrze Multimediów i Grafiki Komputerowej}
\ewykonanaw{the Department of Multimedia and Computer Graphics}
\podkierunkiem{prof. zw. dr hab. Wiesławy Promotorskiej-Profesorskiej}
\epodkierunkiem{prof. zw. dr hab. Wiesława Promotorska-Profesorska}
\end{verbatim}

Klasa dokumentu \texttt{uwmthesis} przyjmuje następujące opcje:
\begin{enumerate}
 \item Nazwa wydziału. W tej chwili wspierany jest tylko jeden wydział, \texttt{wmii}, Wydział Matematyki i~Informatyki. Sposoby wykorzystania klasy dla innych wydziałów opisane są w~rozdziale~\ref{inne}.
 \item Kierunek studiów:
 \begin{description}
  \item [\texttt{inf}]--- informatyka,
  \item [\texttt{mat}]--- matematyka,
  \item [\texttt{fiz}]--- fizyka techniczna.
 \end{description}
\item Rodzaj pracy (poziom studiów):
 \begin{description}
  \item [\texttt{mgr}]--- magisterska,
  \item [\texttt{inz}]--- inżynierska,
  \item [\texttt{lic}]--- licencjacka.
 \end{description}
\item \texttt{twoside}~--- do wydruku dwustronnego.
\end{enumerate}

\smallbreak
W następujących obowiązkowych makrach określa się
\begin{itemize}
 \item Rok złożenia pracy~--- \verb|\date{}|
 \item Tytuł pracy~--- \verb|\title{}|
 \item Tytuł pracy w~języku angielskim~--- \verb|\etitle{}|
 \item Imię i nazwisko autora pracy~--- \verb|\author{}|
 \item Nazwa jednostki, w~której praca została wykonana (w~miejscowniku)~--- \verb|\wykonanaw{}|
 \item Nazwa jednostki, w~której praca została wykonana w języku angielskim~--- \verb|\ewykonanaw{}|
 \item Tytuł, stopień naukowy a~także imię i~nazwisko promotora pracy (w~dopełniaczu)~--- \verb|\podkierunkiem{}|
 \item Tytuł, stopień naukowy a~także imię i~nazwisko promotora pracy w~języku angielskim~--- \verb|\epodkierunkiem{}| Zwróćmy uwagę, że w~pracy~\cite{Maslowski} zaleca się nie tłumaczyć tytułów i~stopni naukowych.
\end{itemize}


