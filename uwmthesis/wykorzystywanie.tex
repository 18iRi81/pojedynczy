\chapter{Wykorzystanie klasy \texttt{uwmthesis}}
\section{Preambuła}

Typowa preambuła wygląda następująco:
\begin{verbatim}
\documentclass[wmii,inf,mgr]{uwmthesis}

\date{2015}
\title{Wykorzystanie Klasy Dokumentu 
  do Pisania Prac Dyplomowych}
\author{Aleksander Denisiuk}
\etitle{Using of the Document Class for Theses}
\wykonanaw{katedrze Multimediów i Grafiki Komputerowej}
\ewykonanaw{the Department of Multimedia and Computer Graphics}
\podkierunkiem{prof. zw. dr hab. Wiesławy Promotorskiej-Profesorskiej}
\epodkierunkiem{prof. zw. dr hab. Wiesława Promotorska-Profesorska}
\end{verbatim}

Klasa dokumentu \texttt{uwmthesis} przyjmuje następujące opcje:
\begin{enumerate}
 \item Nazwa wydziału. W tej chwili wspierany jest tylko jeden wydział, \texttt{wmii}, Wydział Matematyki i~Informatyki. Sposoby wykorzystania klasy dla innych wydziałów opisane są w~rozdziale~\ref{inne}.
 \item Kierunek studiów:
 \begin{description}
  \item [\texttt{inf}]--- informatyka,
  \item [\texttt{mat}]--- matematyka,
  \item [\texttt{fiz}]--- fizyka techniczna.
 \end{description}
\item Rodzaj pracy (poziom studiów):
 \begin{description}
  \item [\texttt{mgr}]--- magisterska,
  \item [\texttt{inz}]--- inżynierska,
  \item [\texttt{lic}]--- licencjacka.
 \end{description}
\item \texttt{twoside}~--- do wydruku dwustronnego.
\end{enumerate}

\smallbreak
W następujących obowiązkowych makrach określa się
\begin{itemize}
 \item Rok złożenia pracy~--- \verb|\date{}|
 \item Tytuł pracy~--- \verb|\title{}|
 \item Tytuł pracy w~języku angielskim~--- \verb|\etitle{}|
 \item Imię i nazwisko autora pracy~--- \verb|\author{}|
 \item Nazwa jednostki, w~której praca została wykonana (w~miejscowniku)~--- \verb|\wykonanaw{}|
 \item Nazwa jednostki, w~której praca została wykonana w języku angielskim~--- \verb|\ewykonanaw{}|
 \item Tytuł, stopień naukowy a~także imię i~nazwisko promotora pracy (w~dopełniaczu)~--- \verb|\podkierunkiem{}|
 \item Tytuł, stopień naukowy a~także imię i~nazwisko promotora pracy w~języku angielskim~--- \verb|\epodkierunkiem{}| Zwróćmy uwagę, że w~pracy~\cite{Maslowski} zaleca się nie tłumaczyć tytułów i~stopni naukowych.
\end{itemize}



\section{Właściwy dokument}
Właściwy dokument pisze się zgodnie z~ogólnymi zasadami pisania prac w~\LaTeX u. Największą jednostką podziału logicznego tekstu jest \verb|\chapter|.

\section{Spis treści}
Spis treści umieszczamy tuż po stronie tytułowej. W taki sposób, początek dokumentu wygląda następująco:
\begin{verbatim}
\begin{document}
\maketitle
\tableofcontents
\end{verbatim}


\section{Wstęp}
Każda praca zaczyna się od wstępu. Zazwyczaj rozdziału ,,Wstęp'' się nie numeruje. Odpowiedni kod jest następujący:
\begin{verbatim}
\chapter*{Wstęp}
\end{verbatim}

\section{Bibliografia}
Zgodnie z~zaleceniami książki~\cite{Osuchowska}, bibliografię umieszczamy po tekście pracy. 

W wymogach~\cite{UWM} nie mówi się o formatowaniu bibliografii. Więc można wprowadzić bibliografię w~zwykły sposób. Na przykład,
\begin{verbatim}
\thebibliography{1}
\bibitem{Oetker}
  Tobias Oetiker et al., \textit{Nie za krótkie wprowadzenie
  do systemu \LaTeXe} 2007. 
  \url{http://www.ctan.org/tex-archive/info/lshort/polish/}
\bibitem{Osuchowska}
  Barbara Osuchowska, \textit{Poradnik redaktora i autora.
  Nauki ścisłe i~technika,} Wydawnictwo PTWK, Warszawa 1988.
\bibitem{UWM}
  Uniwersytet Warmińsko-Mazurski. Wydział Matematyki
  i~Informatyki, \textit{Wytyczne dotyczące pracy dyplomowej,} 2009. 
\bibitem{Wolinski}
  Marcin Woliński, \textit{Moje własne klasy dokumentów dla \LaTeXe.
  Podręcznik użytkownika,} 2009. \url{http://marcinwolinski.pl/mwcls.html}
\end{verbatim}

Jeżeli chcemy zgrupować literaturę według roczników, używamy polecenia \verb|\bibsep{}|. Powyższy przykład będzie wyglądał w taki sposób:
\begin{verbatim}
\thebibliography{1}
\bibsep{1988}
\bibitem{Osuchowska}
  Barbara Osuchowska, \textit{Poradnik redaktora i autora.
  Nauki ścisłe i~technika,} Wydawnictwo PTWK, Warszawa.
\bibsep{2007}
\bibitem{Oetker}
  Tobias Oetiker et al., \textit{Nie za krótkie wprowadzenie
  do systemu \LaTeXe}
  \url{http://www.ctan.org/tex-archive/info/lshort/polish/}
\bibsep{2009}
\bibitem{UWM}
  Uniwersytet Warmińsko-Mazurski. Wydział Matematyki
  i~Informatyki, \textit{Wytyczne dotyczące pracy dyplomowej.}
\bibitem{Wolinski}
  Marcin Woliński, \textit{Moje własne klasy dokumentów dla \LaTeXe.
  Podręcznik użytkownika,} \url{http://marcinwolinski.pl/mwcls.html}
\end{verbatim}
W tym przypadku w~każdej pozycji nie podajemy roku.


Należy zwrócić uwagę, że źródła internetowe także powinny mieć autora, nazwę i~datę wydania. Adresy internetowe wygodnie jest wprowadzać za pomocą polecenia \verb|\url{}| z~pakietu~\texttt{url}. Konieczne jest także wskazanie protokołu (\texttt{ftp://}, \texttt{http://}, \texttt{https://}, etc).

Wikipedia jest źródłem wtórnym. Należy unikać odwoływania się do Wikipedii na rzecz źródeł pierwotnych. Przykładowo, zamiast
\begin{verbatim}
\bibitem{css}
  Wikipedia, \tetxit{Kaskadowe arkusze stylów},
  \url{http://pl.wikipedia.org/wiki/Kaskadowe_arkusze_styl%C3%B3w}
\end{verbatim}
należy użyć
\begin{verbatim}
\bibitem{css}
  W3C, \tetxit{Cascading Style Sheets, level 1}, 2008,
  \url{http://www.w3.org/TR/REC-CSS1/}
\end{verbatim}

\section{Podziękowania}
Rozdział',,Podziękowania'' jest opcjonalny. 
W~razie potrzeby umieszczamy go w~nienumerowanym rozdziale (\verb|\chapter*|) po bibliografii. Na naszym wydziale wszyscy dyplomanci są wdzięczni swoim promotorom za \emph{nieocenioną opiekę merytoryczną i~wskazówki, bez których\dots} Dlatego w~rozdziale ,,Podziękowania'' zazwyczaj nie wymienia się promotora.

\section{Spisy rysunków, tabel, etc}
Spisy rysunków, tabel, przyjętych oznaczeń, etc umieszczamy po rozdziale ,,Podziękowania'' (po bibliografii).

\section{Streszczenia}
Praca powinna zawierać streszczenia w~dwóch językach: polskim i~angielskim.  Należy je umieścić po spisach rysunków, tabel, etc. Dla streszczeń zostały określone dwa środowiska: \texttt{streszczenie}~--- dla streszczenia w~języku polskim oraz \texttt{abstract}~--- dla streszczenia w~języku angielskim.

Przykład streszczeń:
\begin{verbatim}
\begin{streszczenie}
Celem pracy jest opisanie klasy dokumentów \texttt{uwmthesis},
która pozwala na automatyczne generowanie prac dyplomowych,
zgodnych z~wymogami, przyjętymi na Wydziale Matematyki
i~Informatyki UWM w Olsztynie.
\end{streszczenie}

\begin{abstract}
The purpose of this document is to present the \texttt{uwmthesis}
\LaTeX\ document class. This document class can be used for generation
of theses on the Faculty of Mathematics and Computer Science of
the UWM in Olsztyn.
\end{abstract}
\end{verbatim}


\section{Załączniki}
Załączniki (o ile są) umieszczamy na samym końcu dokumentu.

\section{Ogólne wskazówki}
\subsection{Modularyzacja}
Przy napisaniu pracy szczególnie zaleca się umieszczenie każdego rozdziału w~oddzielnym pliku. Plik główny (bez preambuły), w~taki sposób, wyglądał by następująco:
\begin{verbatim}
\begin{document}
  \maketitle
  \tableofcontents
  \input{wstep.tex}
  \input{section1.tex}
  \input{section2.tex}
  \input{section3.tex}
  \begin{streszczenie}
Celem pracy jest opisanie klasy dokumentów \texttt{uwmthesis}, która pozwala na automatyczne generowanie prac dyplomowych, zgodnych z~wymogami, przyjętymi na Wydziale Matematyki i~Informatyki UWM w Olsztynie.
\end{streszczenie}

\begin{abstract}
The purpose of this document is to present the \texttt{uwmthesis} \LaTeX\ document class. This document class can be used for generation of theses on the Faculty of Mathematics and Computer Science of the UWM in Olsztyn.
\end{abstract}
  \input{absract.tex}
  \input{bibliography.tex}
\end{document}
\end{verbatim}

Pliki, zawierające poszczególne składowe pracy kończymy poleceniem~\verb|\endinput|.


W powyższym przykładzie można także przy napisaniu drugiego rozdziału wykomentować pozostałe części.

\subsection{Wykorzystanie klasy na innych wydziałach}\label{inne}
Najlepszym rozwiązaniem jest napisanie do mnie maila z~prośbą o~dodanie wydziału oraz (lub) kierunku studiów. 

Innym sposobem jest dodanie do preambuły kodu:
\begin{verbatim}
\makeatletter
\renewcommand{\wydzi@l}{Wydział Bioinżynierii Zwierząt}
\renewcommand{\kierun@k}{Zootechnika}
\renewcommand{\ewydzi@l}{Faculty of Animal Bioengineering}
\renewcommand{\ekierun@k}{Zootechny}
\makeatother
\end{verbatim}
Zaletą pierwszego sposobu jest, oczywiście, to, że nowa wersja klasy dokumentów zostanie dostępna dla wszystkich.

Dobrym pomysłem jest także zrobić \emph{pull request} na
  \url{https://bitbucket.org/adenisiuk/uwmthesis/}