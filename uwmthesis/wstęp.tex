
\chapter*{Wstęp}

Klasa dokumentów \texttt{uwmthesis} powstała z~myślą dać studentom Wydziału Matematyki i~Informatyki UWM wygodne narzędzie do układania prac dyplomowych. Zakłada się, że użytkownicy nie mają chęci, ani wystarczającej wiedzy dotyczącej składania tekstów, tym nie mniej zależy im na przyzwoitym wyglądzie pracy.


Klasa \texttt{uwmthesis} została oparta o~klasę \texttt{mwbk} Marcina Wolińskiego~\cite{Wolinski}, która, z~kolei, jest spolonizowaną wersją klasy standardowej \texttt{book}. W~klasie \texttt{uwmthesis} zostało określono kilka makr, niezbędnych do wygenerowania książki pracy dyplomowej, spełniającej wymagania~\cite{UWM}. Klasy \texttt{uwmthesis}, zarówno jak i~\texttt{mwbk} są w~rozbudowie. To oznacza, między innym, że formatowanie dokumentu z~nową wersją klasy może się zmienić. Jednak zawsze będzie zachowana zgodność z~aktualnymi wymogami na prace dyplomowe.

Klasa~\texttt{mwbk} jest częścią każdej współczesnej popularnej dystrybucji \TeX a, a~więc powinna już być dostępna na komputerze użytkownika. Na przypadek, gdyby to było nie tak, do dystrybucji dołączone są dwa pliki, \texttt{mwbk.cls} oraz \texttt{mwbk12.clo} z~paczki Marcina Wolińskiego, potrzebne dla~\texttt{uwmthesis}. Całość paczki \texttt{mwcls} można legalnie pobrać z~\url{http://marcinwolinski.pl/mwcls.html}.

Celem tego dokumentu jest przedstawienie sposobu wykorzystania klasy~\texttt{uwmthesis} do pisania pracy dyplomowej. Jednocześnie sam dokument został napisany z~użyciem tej klasy i~może być wykorzystany jako szablon dla pracy swojej.

Celem tego dokumentu nie jest wprowadzenie do systemu \TeX. O podstawach \LaTeX a można dowiedzieć się, na przykład, z~\cite{Oetker}.

Zazwyczaj pracę się drukuje jednostronnie na standardowych kartkach rozmiaru~a4. W~szczególnych przypadkach, do wydruku dwustronnego, należy użyć opcji klasy dokumentu~\texttt{twoside}.


Niniejsza praca składa się z~trzech rozdziałów. W pierwszym opisane są podstawowe zasady wykorzystania klasy do pisania prac dyplomowych. W drugim rozdziale zostały przedstawione krótkie wnioski oraz pomysły na dalszą rozbudowę tej klasy.
Ostatni rozdział zawiera historię zmian dokumentu.

Najnowsza wersja tego dokumentu oraz klasy znajduje się na stronie $$\hbox{\url{http://wmii.uwm.edu.pl/~denisjuk/uwm/latex/}}$$

Repozytorium z projektem klasy \texttt{uwmthesis}:
  $$\hbox{\url{https://bitbucket.org/adenisiuk/uwmthesis/}}$$

Dystrybucja oraz redystrybucja klasy odbywa się na licencji \emph{LaTeX Project Public License,} wersji 1.2 lub późniejszej.

